\documentclass[11pt]{article}

\usepackage{amsmath}
\usepackage{textcomp}

% Add other packages here %


% Put your group number and names in the author field %
\title{\bf Excercise 4\\ Implementing a centralized agent}
\author{Group \textnumero 3 : Robin Clerc, Pierre Vigier}


% N.B.: The report should not be longer than 3 pages %


\begin{document}
\maketitle

\section{Solution Representation}

\subsection{Variables}
% Describe the variables used in your solution representation %
Our solution is described by a $Hashmap$ of capacity $Nb_{Vehicles}$ mapping each vehicle to the ordered list of its objects $MyTask$, its actions.

One $MyTask$ object is described by
\begin{itemize}
    \item a Logist $Task$
    \item the information whether it is a pickup or a delivery
\end{itemize}
There are $2 \times Nb_{Tasks}$ such task to dispatch on the vehicles set. We can remark that one of those actions represents several actual actions for the agent. This representation is far more convenient to deal with constraints.

\subsection{Constraints}
% Describe the constraints in your solution representation %

By constructing the neighbors of a plan, we are going to generate a lot of invalid plans that we should detect to keep only the valid ones to chose the best neighbor.

The constraints to get a valid plan are :
\begin{itemize}
    \item At each time the load of each vehicle cannot be greater than its capacity.
    \item A vehicle can pickup a task only if it did not have already been taken by a vehicle
    \item A vehicle can deliver a task only if already carries the task to deliver.
    \item All the tasks must be delivered at the end : the sum of all $MyTasks$ cumulated by all the vehicles must be $2 \times Nb_{Tasks} MyTasks$
\end{itemize}


\subsection{Objective function}
% Describe the function that you optimize %

In this problem we have a fixed set of tasks, thus a fixed global award and want to generate the maximum profit from this situation : we have to reduce the cost of the plan.

The cost of the plan is defined by the sum of the costs of the plans of all the vehicles : $cost = \sum\limits_{k=1}^Nb_{Vehicles} cost(k^{th}_{Vehicle}) $

%With an infinite number of task we could also choose to minimize the duration of the whole delivery to maximize the profitability

\section{Stochastic optimization}

\subsection{Initial solution}
% Describe how you generate the initial solution %

Our initial solution is obtained by giving all the tasks to the vehicle with the highest capacity.

\subsection{Generating neighbours}
% Describe how you generate neighbors %
To generate neighbors we have several operations :
\begin{itemize}
    \item The IntraChanges : We change the order of the tasks inside a vehicle. The vehicle has a number of tasks $n$ and we select randomly a number $m$ between $0$ and $n$ to choose the number of swaps. Then for each swap we randomly the two $MyTask$ objects to swap.
    \item The InterChanges : For each vehicle we try to give each task (the two $MyTask$ objects associated) to each other vehicle and process IntraChanges in the destination vehicle to select where it fits best.
\end{itemize}

\subsection{Stochastic optimization algorithm}
% Describe your stochastic optimization algorithm %

For each iteration of our algorithm we generate both neighbors by IntraChanges and InterChanges and select the one with the minimal cost. If it is better than the current plan, it becomes the current plan. If it is not better, it also has a probability to become the current plan to avoid getting stuck in local minimum. However we always remember our best result so far and also have a probability to return to this state to avoid getting lost in a dead-end.


\section{Results}

\subsection{Experiment 1: Model parameters}
% if your model has parameters, perform an experiment and analyze the results for different parameter values %

\subsubsection{Setting}
% Describe the settings of your experiment: topology, task configuration, number of tasks, number of vehicles, etc. %
% and the parameters you are analyzing %

\subsubsection{Observations}
% Describe the experimental results and the conclusions you inferred from these results %

\subsection{Experiment 2: Different configurations}
% Run simulations for different configurations of the environment (i.e. different tasks and number of vehicles) %

\subsubsection{Setting}
% Describe the settings of your experiment: topology, task configuration, number of tasks, number of vehicles, etc. %

\subsubsection{Observations}
% Describe the experimental results and the conclusions you inferred from these results %
% Reflect on the fairness of the optimal plans. Observe that optimality requires some vehicles to do more work than others. %
% How does the complexity of your algorithm depend on the number of vehicles and various sizes of the task set? %

\end{document}